\documentclass[smallextended]{svjour3}
%\usepackage{cite}
% *** GRAPHICS RELATED PACKAGES ***
%
%\usepackage[pdftex]{graphicx}
%
\usepackage[cmex10]{amsmath}

% *** ALIGNMENT PACKAGES ***
%
\usepackage{array}
%\usepackage{fixltx2e}

\usepackage{stfloats}
% LaTeX2e). It also provides a command:
\fnbelowfloat
\usepackage{endfloat}
\usepackage{booktabs}
\usepackage{amssymb}
\usepackage{amsmath}
\usepackage{pgf}
\usepackage{tikz}
\usetikzlibrary{arrows,positioning}
\usepackage[hyphens]{url}
\usepackage{color}
\usepackage{graphicx}
\usepackage{xspace}
\newcommand*{\eg}{e.g.\@\xspace}
\newcommand*{\ie}{i.e.\@\xspace}
\newcommand*{\vs}{vs.\@\xspace}

\newcommand*{\EMD}{\mathrm{EMD}}

% correct bad hyphenation here
\hyphenation{wave-let}

\usepackage{color}
\newcommand{\vl}[1]{\textcolor{orange}{Vincent : #1}}
\newcommand{\gl}[1]{\textcolor{red}{Gr\'egoire : #1}}
\newcommand{\ja}[1]{\textcolor{magenta}{Joakim : #1}}
\newcommand{\ml}[1]{\textcolor{blue}{ Mathieu : #1}}

%\usepackage[backend=bibtex,style=numeric-comp,sorting=none,firstinits=true]{biblatex}
%%\bibliography{biblio}
%\renewcommand{\baselinestretch}{2}

\graphicspath{{./}{figures/}}

\begin{document}

\section{Supplementary material}

The asymmetric amplitude profile of Gammatones makes them suitable to model temporal masking in auditory filterbanks~\cite{Fastl2007}.
Yet, the introductory paper on Gammatone wavelets \cite{Venkitaraman2014} does not provide a formula for deducing $\sigma$ from the specification of a quality factor $Q$.
In this appendix, we provide a rationale for choosing the topmost center frequency $\xi$ of a Gammatone wavelet filter bank in a discrete-time setting.
Then, we relate the bandwidth parameter $\sigma$ to the choice of a quality factor Q.
We express both $\xi$ and $\sigma$ in radians, so that converting this value to Hertz would incur a multiplication by $2\pi/f_s$ where $f_s$ is the sampling frequency of the audio signal at hand.


\subsubsection*{Motivation}
Time reversal of a real signal $\boldsymbol{x}(t)$ is equivalent
to the complex conjugation of its Fourier transform $\boldsymbol{\widehat{x}}(\omega$).
As a consequence, the Fourier transform modulus $\vert\boldsymbol{\widehat{x}}(\omega)\vert$
is not only invariant to translation, but also invariant to time reversal.
Yet, although invariance to translation is needed for classification,
invariance to time reversal is an undesirable property.
A simple way to break invariance to time reversal is to choose $\boldsymbol{\psi}(t)$
as an asymmetric wavelet instead of a Gabor symmetric wavelet.

The complex-valued Gammatone wavelet is a modification of the real-valued
Gammatone auditory filter, originated in auditory physiology. The
Gammatone auditory filter of center frequency $\xi$ is defined
as a gamma distribution of order $N\in\mathbb{N}^{*}$ and bandwidth
$\sigma$ modulated by a sine wave, that is,
\[
t^{N-1}\exp(-2\pi\sigma t)\cos(2\pi \xi t).
\]
For a fixed $\sigma$, the integer $N$ controls the relative shape
of the envelope, becoming less skewed as $N$ increases. Psychoacoustical
experiments have shown that, for $N=4$, the Gammatone function provides
a valid approximation of the basilar membrane response in the mammalian
cochlea \cite{Flanagan1960,Patterson1976,Lyon2010}. In particular,
it is asymmetric both in the time domain and in the Fourier domain,
which allows to reproduce the asymmetry of temporal masking as well
as the asymmetry of spectral masking \cite{Fastl2007}. It is thus
used in computational models for auditory physiology \cite{Pressnitzer2005}.
However, it does not comply with the Grossman-Morlet admissibility
condition, because it has a non-negligible average. In addition, because
the Gammatone auditory filter takes real values in the time domain,
its Fourier transform satisfies Hermitian symmetry, which implies
that it does not belong to the space $H^{2}$ of analytic functions.
More generally, there are no real-valued functions in $H^{2}$ \cite{Grossmann1984}.

\subsubsection*{Related work}

With the aim of building a pseudo-analytic admissible Gammatone wavelet,
\cite{Venkitaraman2014} have modified the definition of the Gammatone
auditory filter, by replacing the real-valued sine wave $\cos(2\pi t)$
by its analytic part $\exp(2\pi\mathrm{\mathrm{i}}t)$ and by taking
the first derivative of the gamma distribution, thus ensuring null
mean. The definition of the Gammatone wavelet becomes
\[
\boldsymbol{\psi}(t)=\left(2\pi(\mathrm{i}-\sigma)t^{N-1}+(N-1)t^{N-2}\right)\exp(-2\pi\sigma t)\exp(2\pi\mathrm{i}\xi t)
\]
in the time domain, and
\[
\boldsymbol{\widehat{\psi}}(\omega)=\dfrac{\mathrm{i}\omega\times(N-1)!}{(\sigma+\mathrm{i}(\omega-\xi))^{N}}
\]
in the Fourier domain. Besides its biological plausibility, the Gammatone
wavelet enjoys a near-optimal time-frequency localization with respect
to the Heisenberg uncertainty principle. Furthermore, this time-frequency
localization tends to optimality as $N$ approaches infinity, because
the limit $N\rightarrow+\infty$ yields a Gabor wavelet \cite{Cohen1995}.
Last but not least, the Gammatone wavelet transform of finite order
$N$ is causal, as opposed to the Morlet wavelet transform, which
makes it better suited to real-time applications. From an evolutionary
point of view, it has been argued that the Gammatone reaches a practical
compromise between time-frequency localization and causality constraints \cite{Venkitaraman2014}.


\subsubsection*{Center frequency parameter}

In order to preserve energy and allow for perfect reconstruction,
the Gammatone wavelet filter bank must satisfy the inequalities
\[
1-\varepsilon\leq\vert\boldsymbol{\hat{\phi}}(\omega)\vert+\sum_{\gamma}\vert\boldsymbol{\hat{\psi}}(2^{\gamma}\omega)\vert+\vert\boldsymbol{\hat{\psi}}(-2^{\gamma}\omega)\vert\leq1
\]
for all frequencies $\omega$, where $\varepsilon$ is a small margin
\cite{Anden2014}. Satisfying the equation above near the Nyquist
frequency $\omega=\pi$ can be achieved by placing the log-frequency
$\log_{2}\xi$ of the first (topmost) wavelet in between the log-frequency
$\log_{2}(\xi\times2^{-1/Q})$ of the second wavelet and the log-frequency
$\log_{2}(2\pi-\xi)$ of the mirror of the first wavelet. We obtain
the equation
\[
\log_{2}\xi-\log_{2}(\xi\times2^{-1/Q})=\log_{2}(2\pi-\xi)-\log_{2}\xi,
\]

of which we deduce the identity
\[
\xi=\dfrac{2\pi}{1+2^{1/Q}}.
\]
For $Q=1$, this yields a center frequency of $\xi=\frac{2\pi}{3}$.
For greater values of $Q$, the center frequency $\xi$ tends towards
$\pi$.


\subsubsection*{Bandwidth parameter}

The quality factor $Q$ of the Gammatone wavelet is defined as the
ratio between the center frequency $\xi$ of the wavelet $\boldsymbol{\hat{\psi}}(\omega)$
and its bandwidth $B$ in the Fourier domain. This bandwidth is given
by the difference between the two solutions $\omega$ of the following
equation:

\[
\dfrac{\vert\boldsymbol{\hat{\psi}}(\omega)\vert}{\vert\boldsymbol{\hat{\psi}}(\xi)\vert}=\dfrac{\omega}{\xi}\times\left(1+\dfrac{\left(\omega-\xi\right)^{2}}{\sigma^{2}}\right)^{-N/2}=r,
\]
where the magnitude cutoff $r$ is most
often set to $\sqrt{\frac{1}{2}}$. Let $\Delta\omega=\omega-\xi$.
Raising the above equation to the power $N/2$ yields the following:
\[
\left(1+\dfrac{\Delta\omega}{\xi}\right)^{N/2}=r\times\left(1+\frac{\Delta\omega^{2}}{\sigma^{2}}\right).
\]
Since $\Delta\omega\ll\xi$, we may approximate the left-hand side
with a first-order Taylor expansion. This leads to a quadratic equation
of the variable $\Delta\negmedspace\omega$:
\[
\dfrac{r^{2/N}}{\sigma^{2}}\times\Delta\omega^{2}-\dfrac{2}{N\xi}\times\Delta\omega+\left(r^{2/N}-1\right)=0.
\]
The discriminant of the above equation is:
\[
D=4\times\left(\dfrac{1}{N^{2}\xi^{2}}+\dfrac{r^{2/N}\left(1-r^{2/N}\right)}{\sigma^{2}}\right),
\]
which is a positive number as long as $r<1$. The bandwidth $B$ of
$\boldsymbol{\hat{\psi}}$ is given by the difference between the
two solutions of the quadratic equation, that is:
\[
B=\dfrac{2\sigma^{2}}{r^{2/N}}\times\sqrt{\dfrac{1}{N^{2}\xi^{2}}+\dfrac{r^{2/N}\left(1-r^{2/N}\right)}{\sigma^{2}}}.
\]
Now, let us express the parameter $\sigma$ as a function of some
required bandwidth $B$ at some cutoff threshold $r$. After having
raised the above to its square and rearranged the terms, we obtain
another quadratic equation, yet of the variable $\mbox{\ensuremath{\sigma}}^{2}$:
\[
\dfrac{4}{r^{4/N}N^{2}\xi^{2}}\sigma^{4}+\dfrac{4\times\left(1-r^{2/N}\right)}{r^{2/N}}\sigma^{2}-B^{2}=0
\]
We multiply the equation by $\dfrac{r^{2/N}}{4\times\left(1-r^{2/N}\right)}\neq0:$
\[
\dfrac{1}{r^{2/N}\left(1-r^{2/N}\right)N^{2}\xi^{2}}\sigma^{4}+\sigma^{2}-\dfrac{r^{2/N}B^{2}}{4\times\left(1-r^{2/N}\right)}=0
\]
This leads to defining $\sigma^{2}$ as the unique positive root of
the above polynomial:
\[
\sigma^{2}=\dfrac{r^{2/N}\left(1-r^{2/N}\right)N^{2}\xi^{2}}{2}\times\left(\sqrt{1+\dfrac{B^{2}}{\left(1-r^{2/N}\right)^{2}N^{2}\xi^{2}}}-1\right).
\]
If the filter bank has to be approximately orthogonal, we typically
set $B$ to $B=\left(1-2^{-1/Q}\right)\times\xi$.

We conclude with the following closed form for $\sigma$:
\[
\sigma=K_{N}\times\sqrt{\dfrac{\sqrt{1+h_{N}\left(Q\right)^{2}}-1}{2}}\times\xi,
\]
where
\[
K_{N}=r^{1/N}N\sqrt{1-r^{2/N}}\mbox{ and }h_{N}\left(Q\right)=\dfrac{1-2^{-1/Q}}{N\times\left(1-r^{2/N}\right)}.
\]



% that's all folks
\begin{thebibliography}{10}
\providecommand{\url}[1]{{#1}}
\providecommand{\urlprefix}{URL }
\expandafter\ifx\csname urlstyle\endcsname\relax
  \providecommand{\doi}[1]{DOI~\discretionary{}{}{}#1}\else
  \providecommand{\doi}{DOI~\discretionary{}{}{}\begingroup
  \urlstyle{rm}\Url}\fi


\bibitem{Anden2014}
And{\'e}n, J., Mallat, S.: Deep scattering spectrum.
\newblock IEEE Transactions on Signal Processing \textbf{62}(16), 4114--4128
  (2014)

\bibitem{Cohen1995}
Cohen, L.: Time-frequency analysis.
\newblock Prentice Hall (1995)

\bibitem{Flanagan1960}
Flanagan, J.: Models for approximating basilar membrane displacement.
\newblock Journal of the Acoustical Society of America \textbf{32}(7), 937--937
  (1960)

\bibitem{Grossmann1984}
Grossmann, A., Morlet, J.: {Decomposition of Hardy functions into square
  integrable wavelets of constant shape}.
\newblock SIAM Journal on Mathematical Analysis \textbf{15}(4), 723--736 (1984)


\bibitem{Lyon2010}
Lyon, R., Katsiamis, A., Drakakis, E.: History and future of auditory filter
  models.
\newblock In: Proceedings of the IEEE International Symposium on Circuits and
  Systems (ISCS) (2010)

\bibitem{Patterson1976}
Patterson, R.: Auditory filter shapes derived with noise stimuli.
\newblock Journal of the Acoustical Society of America \textbf{59}(3), 640--654
  (1976)

\bibitem{Pressnitzer2005}
Pressnitzer, D., Gnansia, D.: Real-time auditory models.
\newblock In: Proceedings of the International Computer Music Conference (ICMC)
  (2005)


\bibitem{Venkitaraman2014}
Venkitaraman, A., Adiga, A., Seelamantula, C.S.: {Auditory-motivated Gammatone
  wavelet transform}.
\newblock Signal Processing \textbf{94}, 608--619 (2014)


\bibitem{Fastl2007}
Zwicker, E., Fastl, H.: Psychoacoustics: Facts and models, vol.~22.
\newblock Springer Science \& Business Media (2013)







\end{thebibliography}

\end{document}
